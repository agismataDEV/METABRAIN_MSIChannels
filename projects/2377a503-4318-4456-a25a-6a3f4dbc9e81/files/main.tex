\documentclass{article}
\usepackage{graphicx} % Required for inserting images

\usepackage{amsmath,amssymb}
\usepackage{lipsum} % For dummy text
\usepackage[a4paper,margin=2.54cm]{geometry}  % No margins
\usepackage{graphicx}
\usepackage[absolute,overlay]{textpos}     % For absolute positioning

\usepackage{pdflscape}
\usepackage{array}
\usepackage{geometry}

\renewcommand{\arraystretch}{1.5} % increase row height

% HIGHLIGH PACKAGES
\usepackage{soul}
\sethlcolor{yellow}
\usepackage{etoolbox}
\robustify\cite
\setlength{\parindent}{0pt}   % No indentation
\setlength{\parskip}{1em}     % Space between paragraphs (~one line)

% \usepackage[backend=biber,style=ieee]{biblatex} % or any style you like
% \addbibresource{Ultrasound_Neuromodulation.bib}

\usepackage[numbers]{natbib}
\bibliographystyle{ieeetr}

\title{Ultrasound-Induced Activation of Mechanosensitive Ion Channels via Membrane Deformation}

\begin{document}
\maketitle
% ============================================================
\section{Introduction}
% ============================================================

Low-intensity ultrasound (US) has emerged as a powerful modality for noninvasive neuromodulation and cellular stimulation. A growing body of experimental and theoretical work supports the hypothesis that US acts primarily through mechanical interactions with the cell membrane rather than direct thermal or electrical effects .

Acoustic radiation force, oscillatory pressure fields, and acoustic streaming induce membrane displacement, strain, and tension. These mechanical perturbations activate mechanosensitive ion channels (MSCs), which transduce membrane deformation into ionic currents. This conversion from mechanical to electrical energy constitutes the primary biophysical mechanism underlying ultrasound neuromodulation in neurons and other excitable cells.

At the molecular level, mechanosensitive channels respond either directly to bilayer tension (``force-from-lipid'' mechanism) or indirectly via cytoskeletal tethers and Ca$^{2+}$-dependent signaling pathways \cite{wuTouchTensionTransduction2017}. The present model integrates nonlinear membrane mechanics with ion channel dynamics to capture both direct and indirect mechanotransduction pathways.
% ============================================================
\section{Mechanosensitive Channel Families Considered}
% ============================================================
\subsection{Mechanosensitive Channel Open Probability}
Mechanosensitive channel opening probability is computed using an area-expansion
Boltzmann formulation:
\[
P_0(t) =
\frac{1}{1 + \exp\left[(T_{1/2} - T)\frac{A_{\mathrm{gate}}}{k_B T}\right]}
\left(1 - e^{-T/T_{\mathrm{inact}}}\right)
\label{eq:MSC_Prob}
\]

\subsection{K2P Potassium Channels (TRAAK, TREK-1, TREK-2)}

Two-pore-domain potassium (K2P) channels are among the best-characterized
force-from-lipid mechanosensitive channels that contribute to action potential propagation, sensory transduction, and muscle contraction. Structural and functional studies demonstrate that TRAAK, TREK-1, and TREK-2 open in response to increased in-plane membrane tension without requiring auxiliary proteins or cytoskeletal coupling \cite{brohawnMechanosensitivityMediatedDirectly2014,sorumUltrasoundActivatesMechanosensitive2021, sorumTensionActivationMechanosensitive2024}.

\textbf{Activation mechanism and kinetics}
K2P channels such as TRAAK, TREK-1, and TREK-2 are activated directly by membrane
tension via a force-from-lipid mechanism. Channel opening occurs when increased
in-plane membrane tension stabilizes the open conformation by expanding the
effective gate area embedded in the lipid bilayer. Their gating is regulated by mechanical perturbation of the cell membrane as well as polyunsaturated fatty acids, other lipids, and temperature. \cite{brohawnMechanosensitivityMediatedDirectly2014}

Activation is rapid, occurring on sub-millisecond to millisecond timescales
($\sim 0.1$--$1$ ms) following the onset of membrane stretch. Importantly, K2P
channels do not exhibit intrinsic inactivation on physiological timescales.
Instead, deactivation occurs promptly when membrane tension falls below the
activation threshold, with closure times comparable to activation times
\cite{sorumUltrasoundActivatesMechanosensitive2021}. A consequence of the rapid ultrasonic activation of TRAAK is that even brief stimulation can activate large currents: 0.15~ms and 0.8~ms stimulation result in approximately 50\% and 95\% of the maximal TRAAK current, respectively~\cite{sorumUltrasoundActivatesMechanosensitive2021}.

\textbf{Physiological role}
The K2P current is a fast, non-inactivating, mechanosensitive outward K$^+$ current directly controlled by membrane tension. Upon activation, K2P channels drive the membrane potential toward $E_K$, thereby stabilizing and repolarizing the membrane and providing rapid negative feedback during ultrasound-induced depolarization.

\textbf{Ions conducted}
Primarily K$^+$.

\textbf{Modeling formulation}  
\[
I_{\mathrm{K2P}} = g_{\mathrm{K2P}} \, P_0(t) \, (V_m - E_K2P)
\]
where, $I_{\mathrm{K2P}}$ is the current carried by mechanosensitive two-pore-domain potassium (K2P) channels (e.g., TRAAK, TREK-1, TREK-2). $g_{\mathrm{K2P}}$ is the maximal K2P conductance per unit membrane area (S\,m$^{-2}$). $P_0(t)$ is the time-dependent open probability of the channel, determined by membrane tension induced by ultrasound-driven deformation. $V_m$ is the membrane potential (mV) and $E_K2P$ is the reversal potential. 

\textbf{Parameter justification}
Values for these parameters were chosen based on previous research studies.
$E_K2P$ was given a value of equal to the reversal potential of the Potassium reversal (Nernst) potential (mV). Sorum et al. \cite{sorumUltrasoundActivatesMechanosensitive2021} mention a reverse potential close to $E_{K^+}$, with a value ~ -75 mV for TRAAK. Patel et al. \cite{patelMammalianTwoPore1998} mention a reversal potential at the predicted value of K+ equillibrium potential at -83 mV. It is assumed in our study that it is equal to the Potassium Nerst potential.

The open probability is computed from Eq.~\ref{eq:MSC_Prob} and Sorum et al. \cite{sorumTensionActivationMechanosensitive2024} found that $T_{1/2}$ is 4.4 mN/m , 6.4 mN/m and 5.8 mN/m and the slope is 1.7, 2.3 and 1.4 while the change in area is 2.4 nm$^2$, 1.8 nm$^2$, 2.9 nm$^2$ for TRAAK, TREK-1 and TREK-2, respectively.

Unitary conductances were found to be 73~pS~\cite{sorumTensionActivationMechanosensitive2024} and 65~pS at 100~mV~\cite{blinMixingMatchingTREK2016} for TRAAK; 48~pS at 50~mV~\cite{patelMammalianTwoPore1998} and 88~pS at 100~mV~\cite{blinMixingMatchingTREK2016} for TREK-1; and 40~pS at 100~mV~\cite{blinMixingMatchingTREK2016} for TREK-2.

\cite{Expression}
The K2P family channels are highly expressed in the brain, particularly in the hippocampus and cortex, and are found in both pyramidal cortical neurons and GABAergic interneurons \cite{djillaniRoleTREK1Health2019}. They are also strongly enriched at nodes of Ranvier, with densities approximately 3,000 times higher than in somatic regions \cite{kandaTREK1TRAAKAre2019}.
\subsection{Piezo1/2 Channels}

\textbf{Activation mechanism and kinetics}
Piezo1/2 are large trimeric mechanosensitive channels that respond
directly to membrane tension and curvature. Increased tension flattens the
intrinsic dome-shaped structure of the channel, lowering the energetic barrier
to opening \cite{costePiezo1Piezo2Are2010,wuTouchTensionTransduction2017}.

Activation is extremely fast, occurring within $\sim 0.1$--$2$ ms after tension
application. Piezo1 channels exhibit pronounced inactivation, with time constants
ranging from $\sim 7$ to $50$ ms depending on isoform, membrane composition, and
ionic conditions \cite{costePiezo1Piezo2Are2010, wuTouchTensionTransduction2017, coxRemovalMechanoprotectiveInfluence2016}. Piezo2 excibit faster inactivation \cite{wuTouchTensionTransduction2017}. Following inactivation, channel
availability is restored on slower timescales once tension is relieved.

\textbf{Physiological role}
The Piezo current is a rapidly activating, mechanosensitive, non-selective cation current that mediates Na$^+$ and Ca$^{2+}$ influx in response to membrane tension. C When membrane tension falls below the activation threshold, Piezo channels rapidly deactivate, terminating inward cation flux. Activation of Piezo channels drives the membrane potential toward $E_{\mathrm{Piezo}}$, producing a strong depolarizing current that initiates intracellular Ca$^{2+}$ signaling and downstream mechanotransductive responses during ultrasound exposure.

\textbf{Ions conducted}
Nonselective cation current dominated by Na$^+$, with significant Ca$^{2+}$
permeability.

\textbf{Modeling formulation}  
Based on the activation/inactivation times the current due to Piezo channels can be modeled as 
\[
I_{\mathrm{Piezo}} = g_{\mathrm{Piezo}} \, P_0(t) \, a^2 h \, (V_m - E_{\mathrm{Piezo}})
\]
where $I_{\mathrm{Piezo}}$ denotes the current carried by mechanosensitive Piezo channels (Piezo1 and Piezo2). $g_{\mathrm{Piezo}}$ is the maximal Piezo conductance per unit membrane area (S\,m$^{-2}$). $P_0(t)$ is the time-dependent open probability determined by membrane tension arising from ultrasound-induced membrane deformation. The variables $a$ and $h$ represent the activation and inactivation gating variables, respectively, with $a^2$ accounting for cooperative activation of the Piezo channel subunits. $V_m$ is the membrane potential (mV), and $E_{\mathrm{Piezo}}$ is the effective reversal potential of Piezo channels (mV). 

The activation and inactivation of Piezo channels are modeled using first-order kinetics:
\[
\frac{d a_{\mathrm{Pz}}}{dt} = \frac{P_0(t) - a_{\mathrm{Pz}}}{\tau_{a, \mathrm{Piezo}}}, \quad
\frac{d h_{\mathrm{Pz}}}{dt} = \frac{1 - h_{\mathrm{Pz}}}{\tau_{h, \mathrm{Piezo}}}
\]
where $\tau_{a, \mathrm{Piezo}} $ is the activation time constant in ms and $\tau_{h, \mathrm{Piezo}} $ is the inactivation time constant in ms (representing recovery to fully available state).

\textbf{Parameter justification}
In this model, $E_{\mathrm{Piezo}}$ was set to $0$~mV, reflecting their non-selective cation permeability. Piezo channels are nonselective cation channels permeable to Na$^+$, K$^+$, and Ca$^{2+}$ and exhibit a near-linear current--voltage relationship with reversal potentials close to 0~mV
\cite{costePiezo1Piezo2Are2010, wuTouchTensionTransduction2017, shinPiezo2IonChannel2019}. 
 
The open probability is computed from Eq.~\ref{eq:MSC_Prob} and Cox et al. \cite{coxRemovalMechanoprotectiveInfluence2016} found that $T_{1/2}$ is 4.5 - 5 mN/m , while the change in area is 8 - 15 nm$^2$. Wu et al. \cite{wuTouchTensionTransduction2017} found that $T_{1/2}$ is 1.4 mN/m , while the change in area is 6 - 20 nm$^2$.

Unitary conductances were found to be 22~pS and 28~pS~\cite{shinPiezo2IonChannel2019} for Piezo1 and Piezo2, respectively.
The channel density was found to be 1-2 channels/ $\mu$m$^2$ \cite{lewisPiezo1IonChannels2024}.

A small fraction of the Piezo current is assumed to contribute to intracellular Ca$^{2+}$ influx, specifically 5–10\% of the total current, to account for the Ca$^{2+}$ permeability of Piezo channels observed experimentally \cite{gnanasambandamIonicSelectivityPermeation2015}. Single-channel recordings of human Piezo1 demonstrate that the channel is non-selective among cations, with a relative permeability sequence for monovalent ions of K$^+ > $Cs$^+ \approx$ Na$^+ >$ Li$^+$ (1.0 : 0.88 : 0.82 : 0.71) and corresponding unitary conductances at negative potentials of approximately 47, 39, 36, and 23 pS, respectively. Divalent ions such as Ca$^{2+}$, Ba$^{2+}$, and Mg$^{2+}$ also permeate the channel, though with smaller unitary conductances (~15 pS for Ca$^{2+}$, 25 pS for Ba$^{2+}$, and 10 pS for Mg$^{2+}$), reflecting slower permeation relative to monovalents.

\cite{Expression}
Piezo channels are broadly expressed in the nervous system, with Piezo1 found in various central and peripheral neurons as well as glial cells, and Piezo2 predominantly in sensory neurons, including dorsal root ganglion (DRG) and trigeminal neurons. Both channels are localized to mechanosensitive sites where they mediate rapid cation influx in response to membrane tension, and their expression patterns allow them to contribute to touch, proprioception, and other mechanosensory processes.
\subsection{TRP Channels (TRPC1, TRPP1, TRPP2)}

Members of the TRP channel family participate in mechanosensitive processes, but for many mammalian TRP isoforms there is limited evidence that they are directly gated by bilayer tension; instead, activation mechanisms can involve tethering to structural proteins, cytoskeletal interactions, or downstream signaling pathways triggered by mechanical stimuli \cite{Christensen2007, nikolaevMammalianTRPIon2019}.
TRPP2 (polycystin‑2) and TRPC1 are non‑selective Ca$^{2+}$‑permeable cation channels implicated in mechanosensitive pathways, but there is limited robust evidence that they act as primary force‑from‑lipid sensors comparable to PIEZO1/2. In many mechanotransduction contexts, TRP channels are proposed to function as mechano‑amplifiers or downstream effectors of primary mechanotransducers (e.g., PIEZO channels) rather than directly gated by membrane tension \cite{CoxPoole2004}.

\textbf{Ions conducted}  
These channels conduct mixed Na$^+$ and Ca$^{2+}$ currents; TRPP2 has measurable Ca$^{2+}$ permeability and TRPC1 contributes to Ca$^{2+}$ entry in native heteromers, but both show modest Ca$^{2+}$ fractions of total current under physiological ionic conditions \cite{pedersenTRPChannelsOverview2005, geesRoleTransientReceptor2010}.


\textbf{Activation and inactivation mechanisms.}  
TRPP2 (polycystin-2) exhibits direct cytosolic Ca$^{2+}$ regulation mediated by a C-terminal EF-hand domain. Single-channel recordings demonstrate that TRPP2 open probability increases as cytosolic free Ca$^{2+}$ rises from low micromolar to tens of micromolar concentrations, reaching maximal activity at submillimolar Ca$^{2+}$, while higher Ca$^{2+}$ concentrations inhibit channel activity, yielding a bell-shaped dependence of open probability on intracellular Ca$^{2+}$ \cite{koulenPolycystin2IntracellularCalcium2002}. The maximal absolute open probability of TRPP2 is low, on the order of 2--3\%, and is further modulated by membrane potential, with more negative potentials increasing channel availability \cite{koulenPolycystin2IntracellularCalcium2002}.

In contrast, TRPC1 does not possess an intrinsic Ca$^{2+}$-activation gate. Its opening is mediated by receptor-operated and store-operated signaling pathways, while elevated intracellular Ca$^{2+}$ produces calmodulin-dependent feedback inhibition, reducing channel open probability as Ca$^{2+}$ accumulates \cite{singhCalmodulinRegulatesCa2+Dependent2002}. Thus, Ca$^{2+}$-dependent inactivation is included for TRPC1, whereas Ca$^{2+}$-dependent activation is not modeled explicitly.

\textbf{Modeling formulation}  
\[
I_{\mathrm{TRP}} = g_{\mathrm{TRP}} \, a_{\mathrm{TRP}} \, h_{\mathrm{TRP}} \, (V_m - E_{\mathrm{TRP}})
\]
where $a_{\mathrm{TRP}}$ and $h_{\mathrm{TRP}}$ gate toward the bell‑shaped steady‑state probabilities, $V_m$ is membrane potential, and $E_{\mathrm{TRP}}$ is the mixed cation reversal potential. This description reproduces experimentally observed depolarizing TRP currents with positive reversal potentials and modest Ca$^{2+}$ influx characteristic of receptor‑operated and Ca$^{2+}$‑modulated TRP channel activity \cite{pedersenTRPChannelsOverview2005,geesRoleTransientReceptor2010}.

TRPP2 open probability is modeled as a product of Ca$^{2+}$-dependent activation and inactivation terms, scaled by a voltage-dependent availability factor:

\[
a_{\mathrm{TRPP1/2}}^{\infty}([Ca]_i) =
\frac{[Ca]_i^{n_a}}{K_a^{n_a}+[Ca]_i^{n_a}},
\quad
h_{\mathrm{TRPP1/2}}^{\infty}([Ca]_i) =
\frac{K_i^{n_i}}{K_i^{n_i}+[Ca]_i^{n_i}},
\]

\[
P_{\mathrm{open}}^{\mathrm{TRPP2}}(Ca_i,V_m)
=
P_{\max}
\underbrace{a_{\mathrm{TRPP1/2}}^{\infty}([Ca]_i)}_{\text{activation}}
\underbrace{h_{\mathrm{TRPP1/2}}^{\infty}([Ca]_i)}_{\text{inactivation}}
f_V(V_m),
\]
with $P_{\max}=0.03$, activation half-max $K_a \sim 0.1$--$0.5~\mu$M, inactivation half-max $K_i \sim 1$--$1.5~\mu$M, and Hill coefficients $n_a,n_i \approx 1$--$2$, based on single-channel measurements \cite{koulenPolycystin2IntracellularCalcium2002}. Although TRPP2 gating is naturally regulated by local microdomain Ca$^{2+}$ near the channel, here bulk cytosolic Ca$^{2+}$ ($Ca_i$) is used for simplicity. The voltage-dependent factor $f_V(V_m)$ increases channel availability at negative membrane potentials and is normalized to unity near $-30$~mV.\cite{koulenPolycystin2IntracellularCalcium2002}

For TRPC1-containing channels, Ca$^{2+}$-dependent inactivation is modeled phenomenologically as

\[
a_{\mathrm{TRPC1}}^{\infty} = P_\mathrm{max} \, P_0^{\mathrm{Piezo1}},
\quad
h_{\mathrm{TRPC1}}^{\infty}([Ca]_i) =
h_{\mathrm{TRPP1/2}}^{\infty}([Ca]_i) ,
\]
with $P_{\max}=0.05$, $K_i \sim 0.3$--$1~\mu$M, reflecting calmodulin-mediated negative feedback \cite{singhCalmodulinRegulatesCa2+Dependent2002}.

The activation and Ca$^{2+}$-dependent inactivation are modeled as:
\[
\frac{d a_{\mathrm{TRP}}}{dt} = \frac{a_{\mathrm{TRP}}^{\infty} - a_{\mathrm{TRP}}}{\tau_{a, \mathrm{TRP}}},
\quad
\frac{d h_{\mathrm{TRP}}}{dt} = \frac{h_{\mathrm{TRP}}^{\infty}(\mathrm{Ca}) - h_{\mathrm{TRP}}}{\tau_{h, \mathrm{TRP}}}
\]
with $\tau_{a, \mathrm{TRP}} = 20$ ms for activation and $\tau_{h, \mathrm{TRP}} = 100$ ms for Ca$^{2+}$-dependent inactivation.

Single-channel recordings from heterologous expression systems have provided estimates of the unitary conductances of TRPP2 and TRPC1 channels. TRPP2 (polycystin-2) channels exhibit relatively large conductance when recorded either in isolation or in association with PKD1, with reported mean unitary conductances on the order of $\approx 140$--$170~\mathrm{pS}$ under symmetrical ionic conditions \cite{pedersenTRPChannelsOverview2005,baiFormationNewReceptoroperated2008}. In contrast, TRPC1 alone displays a much smaller unitary conductance of approximately $\approx 16$--$17~\mathrm{pS}$ \cite{pedersenTRPChannelsOverview2005,baiFormationNewReceptoroperated2008}, consistent with its role as a low-conductance pore-forming subunit that typically contributes to heteromeric TRPC channel complexes. When TRPP2 and TRPC1 are co-assembled, intermediate single-channel conductances near $\approx 40~\mathrm{pS}$ have been reported, indicating that TRPC1 can modulate pore properties and attenuate the high conductance characteristic of TRPP2 homomeric channels \cite{baiFormationNewReceptoroperated2008}. These measurements are consistent with broader surveys of TRP channel biophysics, which report a wide range of single-channel conductances (from $\approx 10~\mathrm{pS}$ to $>100~\mathrm{pS}$) across the TRP family, reflecting differences in pore architecture, subunit composition, and ion permeation properties \cite{geesRoleTransientReceptor2010}.

Electrophysiological recordings of TRPP2 and TRPC1 channels indicate that both exhibit non‑selective cation I–V relationships with reversal potentials near the equilibrium for mixed monovalent cations, consistent with limited ionic selectivity. For TRPC1, slope conductance measurements in native and heterologous systems show an almost linear current–voltage relationship with an extrapolated reversal potential of approximately +30~mV under standard Na$^{+}$/K$^{+}$ ionic gradients \cite{skopinTRPC1ProteinForms2013}. For TRPP2, direct measurements of reversal potentials for homomeric channels are less frequently reported, but whole‑cell and single‑channel analyses typically display reversal near 0–+10~mV for mixed monovalent cation currents and modest outward rectification, reflecting non‑selective conductance of Na$^{+}$, K$^{+}$ and Ca$^{2+}$ \cite{pedersenTRPChannelsOverview2005,baiFormationNewReceptoroperated2008}. Co‑assembly of TRPP2 with TRPC1 has been shown to produce a positive shift in reversal potential of roughly +8~mV relative to channels expressed alone, indicating altered relative permeabilities in the heteromeric complex \cite{baiFormationNewReceptoroperated2008}. These I–V profiles are consistent with permeability ratios that do not strongly favor any single cation species and with modest contributions of Ca$^{2+}$ to the overall current.

TRPP2 (polycystin-2) and TRPC1 are non-selective cation channels with limited Ca$^{2+}$ permeability, a common feature of TRP channels \cite{pedersenTRPChannelsOverview2005,geesRoleTransientReceptor2010}. For TRPP2, bi-ionic reversal-potential measurements indicate $P_{\mathrm{Ca}}/P_{\mathrm{Na}} \approx 1$--$3$, implying that Ca$^{2+}$ contributes only a small fraction of the total current ($\sim 1s-4\,\%$ under physiological conditions, $\sim$140~mM Na$^{+}$, $\sim$2~mM Ca$^{2+}$) \cite{storchTransientReceptorPotential2012}. In contrast, TRPC1 forms heteromeric channels with reduced Ca$^{2+}$ permeability compared with other TRPC isoforms \cite{storchTransientReceptorPotential2012}, yielding $P_{\mathrm{Ca}}/P_{\mathrm{Na}} < 1$ and an estimated Ca$^{2+}$ fraction of $<1-3\,\%$. These data indicate that TRPP2 and TRPC1 primarily mediate depolarizing Na$^{+}$ influx, with only modest accompanying Ca$^{2+}$ entry.

\textbf{Expression}
TRPC1 is the most widely distributed member of the TRPC subfamily in the mammalian brain and is particularly abundant in cortical pyramidal neurons, where it localizes to somata and apical dendrites, with additional expression reported in subsets of GABAergic interneurons but little to no expression in glial cells (von Bohlen und Halbach et al., 2018; Riccio et al., 2002). In contrast, TRPP family members (TRPP1/PKD1 and TRPP2/PKD2) show lower and more heterogeneous neuronal expression. Transcriptomic and immunohistochemical studies indicate that TRPP2 is present in both neurons and glia, with evidence for neuronal expression in cortical and hippocampal populations, although its functional contribution at the plasma membrane of pyramidal neurons or interneurons remains less well characterized compared to TRPC1 (González-Perrett et al., 2001; Wu et al., 2021). TRPP1 is primarily known for its role as a regulatory subunit forming complexes with TRPP2, and neuronal expression has been reported, but with limited cell-type specificity and sparse electrophysiological characterization in cortical circuits.
\subsection{TRPM4 (Ca$^{2+}$-Activated Nonselective Cation Channel)}

TRPM4 is not directly mechanosensitive but is activated by elevations in
intracellular Ca$^{2+}$ following Piezo and TRP channel opening
\cite{launayTRPM4Ca2+ActivatedNonselective2002, guoStructuresCalciumactivatedNonselective2017}.

\textbf{Activation mechanism and kinetics}
TRPM4 activation depends on intracellular Ca$^{2+}$ concentration. Experimentally,
normalized activation currents (I/I$_{\rm max}$) can be fitted with a Hill equation
\cite{guoStructuresCalciumactivatedNonselective2017}:
\[
P_{\rm TRPM4} = \frac{[{\rm Ca}^{2+}]^n}{[{\rm Ca}^{2+}]^n + EC_{50}^n}
\]
where $n$ is the Hill coefficient (cooperativity), and EC$_{50}$ is the half-activation Ca$^{2+}$ concentration.  

Based on previous patch-clamp studies:  
\begin{itemize}
  \item EC$_{50} \approx 0.3$--1~$\mu$M \cite{launayTRPM4Ca2+ActivatedNonselective2002}, consistent with physiological Ca$^{2+}$ elevations after Piezo/TRP activation.
  \item Hill coefficient $n \approx 4 - 6$\cite{launayTRPM4Ca2+ActivatedNonselective2002}, indicating cooperative Ca$^{2+}$ binding.
\end{itemize}

TRPM4 activation occurs within $\sim 10$--$100$~ms after intracellular Ca$^{2+}$ rises, depending on diffusion and channel density. TRPM4 does not exhibit classical inactivation; the current decreases as intracellular Ca$^{2+}$ returns to baseline due to buffering and extrusion mechanisms. This avoids introducing additional gating variables since the activation is relatively fast (10–100 ms) and TRPM4 does not inactivate.

\textbf{Physiological role}
TRPM4 is a Ca$^{2+}$-activated, monovalent-selective cation channel that provides a depolarizing inward current without directly contributing to Ca$^{2+}$ influx. Although channel opening requires elevations in intracellular Ca$^{2+}$, TRPM4 is effectively impermeable to divalent cations, including Ca$^{2+}$, and conducts primarily monovalent ions, with Na$^+$ dominating the current under physiological conditions due to its large electrochemical driving force \cite{launayTRPM4Ca2+ActivatedNonselective2002}. As a result, TRPM4 acts as an electrical amplifier of upstream Ca$^{2+}$ signals, converting local Ca$^{2+}$ elevations—originating from mechanosensitive channels, voltage-gated Ca$^{2+}$ channels, or intracellular stores—into membrane depolarization. Structural studies confirm that the TRPM4 pore architecture excludes Ca$^{2+}$ permeation while supporting monovalent ion flow, consistent with its role as a Ca$^{2+}$-gated but Ca$^{2+}$-impermeable channel \cite{guoStructuresCalciumactivatedNonselective2017}. Functionally, TRPM4 contributes to action potential initiation and burst firing by prolonging depolarization and increasing neuronal excitability, while influencing intracellular Ca$^{2+}$ dynamics only indirectly through voltage-dependent mechanisms rather than direct Ca$^{2+}$ entry.

\textbf{Modeling formulation}  
In our Hodgkin--Huxley-type framework, the TRPM4 current is modeled as:
\[
I_{\mathrm{TRPM4}} = g_{\mathrm{TRPM4}} \, P_{\rm TRPM4} \, (V_m - E_{\mathrm{TRPM4}})
\]
with the Ca$^{2+}$-dependent open probability given as described before.

\textbf{Parameter justification}
In this model, the reversal potential of TRPM4 ($E_{\mathrm{TRPM4}}$) was set to $0$~mV, consistent with experimental measurements showing that TRPM4 is a Ca$^{2+}$-activated, monovalent non-selective cation channel with approximately equal permeability to Na$^+$ and K$^+$, but impermeable to Ca$^{2+}$ \cite{launayTRPM4Ca2+ActivatedNonselective2002}. As a consequence of its mixed Na$^+$/K$^+$ permeability, the opposing transmembrane gradients of these ions yield a reversal potential close to $0$~mV under physiological ionic conditions. Although TRPM4 is non-selective among monovalent cations, the inward current at resting membrane potentials is dominated by Na$^+$ influx due to its large electrochemical driving force.

Unitary conductances were found to be 25~pS \cite{launayTRPM4Ca2+ActivatedNonselective2002}.

Voltage dependence is negligible for TRPM4 under physiological conditions, so the model includes only Ca$^{2+}$-dependent gating. The fraction of Ca$^{2+}$-activated current is fully directed to Na$^+$ influx, consistent with literature reports \cite{guoStructuresCalciumactivatedNonselective2017}.
\subsection{T-Type Ca$^{2+}$ Channels}

\textbf{Activation mechanism and kinetics}
T-type Ca$^{2+}$ channels are low-voltage-activated, transient calcium channels
that open in response to modest depolarizations. They are recruited following
mechanosensitive channel activation and provide additional Ca$^{2+}$ influx,
amplifying subthreshold depolarizations \cite{huguenardSimulationCurrentsInvolved1992}.

\textbf{Physiological role}
These channels act as amplifiers of subthreshold depolarizations initiated by
mechanosensitive currents. They contribute to intracellular Ca$^{2+}$ dynamics,
which can further activate Ca$^{2+}$-dependent channels such as TRPM4 and
K-Ca channels.


\textbf{Modeling formulation}  
The T-type Ca$^{2+}$ current is modeled as:
\[
I_{\mathrm{T}} = g_T \, a_T(V_m)^2 \, b_T(V_m) \, (V_m - E_{\mathrm{T}})
\]
where, $I_{\mathrm{T}}$ is the T-type Ca$^{2+}$ current (A/m$^2$), $g_T$ is the maximal channel conductance. $a_T(V_m)$ is the steady-state activation probability:
    \[
        a_T(V_m) = \frac{1}{1 + \exp(-(V_m - V_{1/2, \mathrm{act}})/k_\mathrm{act})}.
    \]
$b_T(V_m)$ is the steady-state inactivation probability:
    \[
        b_T(V_m) = \frac{1}{1 + \exp((V_m - V_{1/2, \mathrm{inact}})/k_\mathrm{inact})}.
    \]
$E_{\mathrm{T}}$ is the reversal potential for T-type channels (mV), corresponding to the Nernst potential for Ca$^{2+}$ (mV).

\textbf{Parameter justification}
\begin{itemize}
    \item The steady-state activation parameters of the T-type Ca$^{2+}$ current,
    with a half-activation voltage $V_{1/2, \mathrm{act}} = -57$~mV and slope factor of $k_\mathrm{act} = 5.6$~mV,
    were taken from Huguenard and McCormick (1992) \cite{huguenardSimulationCurrentsInvolved1992},
    consistent with experimentally measured low-threshold Ca$^{2+}$ channel
    activation in thalamic and other central neurons.

    \item The steady-state inactivation parameters, with a half-inactivation
    voltage $V_{1/2, \mathrm{inact}} = -80$~mV and slope factor of $k_\mathrm{inact} = 4.0$~mV, reproduce the rapid
    voltage-dependent inactivation characteristic of T-type Ca$^{2+}$ channels,
    as reported experimentally and implemented in \cite{huguenardSimulationCurrentsInvolved1992}.

    \item Single-channel recordings under physiological ionic conditions report
    a unitary conductance in the range of approximately 5–9~pS for T-type
    Ca$^{2+}$ channels, which was used to guide the choice of maximal conductance
    in the model \cite{huguenardSimulationCurrentsInvolved1992}.

    \item All T-type Ca$^{2+}$ current was included in the intracellular Ca$^{2+}$
    dynamics, as these channels are highly Ca$^{2+}$ selective and represent a
    major source of activity-dependent Ca$^{2+}$ influx. This modeling assumption
    follows the approach of Huguenard and McCormick (1992), where Ca$^{2+}$ entry
    through T-type channels directly contributes to intracellular Ca$^{2+}$
    accumulation and downstream Ca$^{2+}$-dependent processes. \cite{huguenardSimulationCurrentsInvolved1992}
\end{itemize}


\textbf{Activation and inactivation kinetics}
T-type channels activate rapidly upon depolarization and inactivate over tens of milliseconds,
providing transient Ca$^{2+}$ influx. Due to their fast kinetics relative to slower
mechanosensitive currents, explicit gating variables can be omitted in some
simplified models, but in this implementation, steady-state activation and
inactivation probabilities ($a_T$, $b_T$) are used to capture the transient response
accurately.
\subsection{Ca$^{2+}$-Activated K Channels and Repolarization}

Sustained Ca$^{2+}$ entry through Piezo, TRP, and T-type channels necessitates
negative feedback mechanisms to restore the resting membrane potential.
Ca$^{2+}$-activated K$^+$ channels provide this stabilizing outward current \cite{leindersCa2DependenceSmall1992}.

\textbf{Modeled current}
\[
I_{\mathrm{KCa}} = g_{\mathrm{KCa}}
\frac{[\mathrm{Ca}]^n}{[\mathrm{Ca}]^n + K_d^n}
(V_m - E_K),
\]
where the reversal potential is equal to the Potassium Nernst equilibrium potential and \(n = 10\). 
\(K_\mathrm{d}\) for SK channels was found to be 1 $\mu$M of Ca$^{2+}$ \cite{leindersCa2DependenceSmall1992}. Including \(I_{\mathrm{KCa}}\) was essential to prevent pathological depolarization and to recover the physiological resting potential in long simulations.

\subsection{Modeled M-type K$^+$ current with Ca$^{2+}$-dependent modulation}

\textbf{Modeling formulation}  
The M-type potassium current was modeled as a slowly activating, non-inactivating outward K$^+$ current that stabilizes the membrane potential and opposes repetitive firing, with Ca$^{2+}$-dependent modulation of its maximal conductance:
\[
I_{\mathrm{M}} = g_{\mathrm{M,eff}} \, p \, (V_m - E_K), 
\qquad 
g_{\mathrm{M,eff}} = g_{\mathrm{M,bar}} \left(1 - 0.732 \frac{1}{1 + (K_{d,\mathrm{M}}/[\mathrm{Ca}])^n} \right).
\]
Here, \(I_{\mathrm{M}}\) is the M-type K$^+$ current, \(g_{\mathrm{M,bar}}\) is the maximal M-channel conductance per unit membrane area (S\,m$^{-2}$), \(p\) is the voltage-dependent activation variable of the M current, \(V_m\) is the membrane potential, \(E_K\) is the potassium reversal potential, \([\mathrm{Ca}]\) denotes intracellular calcium concentration (\(\mu\)M), \(K_{d,\mathrm{M}}\) is the half-inhibition constant, and \(n\) is the Hill coefficient describing Ca$^{2+}$-dependent suppression of M-channel conductance.

M-type K$^+$ channels (Kv7/KCNQ) are concentrated in the perisomatic region of CA1 pyramidal neurons (but also in cortical neurons), where they regulate synaptic integration, resting membrane potential, and action potential initiation \cite{huMChannelsKv7KCNQ2007}. The effective conductance \(g_{\mathrm{M,eff}}\) decreases as intracellular Ca$^{2+}$ rises, reflecting Ca$^{2+}$/calmodulin-mediated inhibition of Kv7 channels. Experimental measurements show that intracellular Ca$^{2+}$ suppresses KCNQ2/3 currents with a half-maximal inhibition (IC$_{50}$) of approximately 70 nM cite{gamperCalmodulinMediatesCa2+dependent2003}, which was adopted as \(K_{d,\mathrm{M}}\) in the model. The data were fit by a Hill equation with an IC$_{50}$ of 70 nM, a saturating inhibition of 0.732, and a Hill coefficient \(n = 2\) \cite{gamperCalmodulinMediatesCa2+dependent2003}. This Ca$^{2+}$-dependent modulation operates on timescales of tens to hundreds of milliseconds, reflecting Ca$^{2+}$ accumulation and Ca$^{2+}$/calmodulin-mediated regulation of Kv7 channels. It also allows depolarizing currents, such as those carried by Piezo or TRP channels, to transiently overcome stabilizing M-current and enhance excitability. By incorporating this mechanism, the model captures the dynamic interplay between intracellular Ca$^{2+}$ elevations and outward K$^+$ conductances during mechanosensitive or synaptic stimulation. 
The unitary conductance was found to be 10~pS \cite{huMChannelsKv7KCNQ2007}.

\subsection{Intracellular Ca$^{2+}$ Dynamics and Parameter Justification}

Intracellular calcium dynamics were modeled using a reduced first-order balance
between Ca$^{2+}$ influx through Ca$^{2+}$-permeable membrane currents and Ca$^{2+}$ removal by buffering, diffusion, and active extrusion mechanisms. The temporal evolution of intracellular Ca$^{2+}$ concentration is described by \cite{helmchenSingleCompartmentModelCalcium2015}
\[
\frac{d[\mathrm{Ca}^{2+}]}{dt} = - \alpha I_{\mathrm{Ca}} - \frac{[\mathrm{Ca}^{2+}]}{\tau_{\mathrm{Ca}}} ,
\]
where $I_{\mathrm{Ca}}$ is the total Ca$^{2+}$-carrying membrane current density, $\alpha$ is a phenomenological conversion factor linking membrane current to changes in intracellular Ca$^{2+}$ concentration, and $\tau_{\mathrm{Ca}}$ is an effective Ca$^{2+}$ decay time constant representing the combined effects of buffering, diffusion, and extrusion.

This reduced formulation is obtained under the following assumptions:
(i) intracellular Ca$^{2+}$ is spatially homogeneous within the modeled
compartment, such that spatial gradients and diffusion can be lumped into a single effective decay term; (ii) Ca$^{2+}$ transients remain small compared to buffer dissociation constants, allowing buffering dynamics to be treated as fast and linear and eliminating the need to explicitly model buffer-bound Ca$^{2+}$; (iii) Ca$^{2+}$ extrusion mechanisms operate in a near-linear regime around the resting Ca$^{2+}$ concentration, so that nonlinear pump and exchanger kinetics can be approximated by a first-order relaxation with time constant $\tau_{\mathrm{Ca}}$; (iv) Ca$^{2+}$ influx is dominated by membrane Ca$^{2+}$ currents during action potentials, with other sources such as intracellular stores neglected; and (v) the resting Ca$^{2+}$ concentration is absorbed into the definition of the state variable or taken as zero without loss of generality. \cite{helmchenSingleCompartmentModelCalcium2015}

\textbf{Ca$^{2+}$ decay time constant.}
The Ca$^{2+}$ decay time constant was set to $\tau_{\mathrm{Ca}} = 50$ ms, which lies well within experimentally observed ranges for cortical pyramidal neurons. Two-photon calcium imaging studies report that somatic and hCa$^{2+}$ transients typically decay on timescales of tens of milliseconds, with reported values ranging from approximately 10 to 100 ms depending on compartment, buffer load, and pump activity.\cite{helmchenSingleCompartmentModelCalcium2015}

Helmchen et al.\ \cite{helmchenCa2BufferingAction1996} showed that Ca$^{2+}$ signals in pyramidal neuron dendrites exhibit “decay time constants of several tens of milliseconds,” while Svoboda et al.\cite{svobodaVivoDendriticCalcium1997} described Ca$^{2+}$ transients that “relax back to baseline within tens of milliseconds.” Sabatini et al. \cite{sabatiniLifeCycleCa22002} further demonstrated that endogenous buffering and extrusion mechanisms shape Ca$^{2+}$ signals with decay constants spanning this same range. More recent reviews confirm that effective Ca$^{2+}$ clearance in cortical neurons is typically “on the order of tens of milliseconds”.

Thus, a value of 50 ms represents a physiologically reasonable midpoint and is commonly employed in reduced conductance-based neuron models to capture the net effect of calcium removal mechanisms without resolving spatial diffusion explicitly \cite{destexheMechanismsUnderlyingSynchronizing1998}.

\textbf{Current-to-calcium coupling factor.}
The parameter $\alpha$ scales the contribution of Ca$^{2+}$ influx to intracellular Ca$^{2+}$ concentration changes and implicitly incorporates Faraday conversion, effective cytosolic volume, and buffering. In the present model,
$\alpha = 0.01$ was selected as a phenomenological coupling coefficient that produces Ca$^{2+}$ transients on the order of $0.1$--$1~\mu$M in response to spike-evoked and mechanically induced Ca$^{2+}$ currents.

Experimental measurements indicate that resting intracellular Ca$^{2+}$ in cortical and hippocampal pyramidal neurons is maintained in the tens of nanomolar range. For example, Helmchen et al.~\cite{helmchenCa2BufferingAction1996} reported a mean resting Ca$^{2+}$ concentration of $64 \pm 4$~nM, while other imaging studies report baseline concentrations of approximately $50$--$100$~nM depending on cellular compartment and buffering conditions \cite{svobodaVivoDendriticCalcium1997,sabatiniLifeCycleCa22002}. Single action potentials typically evoke transient Ca$^{2+}$ elevations of $\sim 0.1$--$0.5~\mu$M, whereas brief spike bursts can raise intracellular Ca$^{2+}$ into the $0.5$--$1~\mu$M range \cite{helmchenCa2BufferingAction1996,sabatiniLifeCycleCa22002}. In addition, Ca$^{2+}$ concentrations exceeding $1~\mu$M have been recorded, particularly within submembrane or nanodomain microenvironments near open Ca$^{2+}$ channels during strong synaptic activity, as well as under excessive or pathological conditions such as sustained depolarization, ischemia, or excitotoxic stress \cite{bootmanFundamentalsCellularCalcium2020}.

\textbf{Physiological consistency.}
With these parameter choices, resting intracellular Ca$^{2+}$ remains near $0.05$--$0.1~\mu$M, while electrical or mechanical stimulation induces transient Ca$^{2+}$ elevations of $0.1$--$1~\mu$M that decay on a tens-of-milliseconds timescale. Higher Ca$^{2+}$ levels ($>1~\mu$M) are interpreted as localized microdomain signals or indicators of excessive stimulation rather than steady-state cytosolic concentrations. This behavior is consistent with in vivo and in vitro calcium imaging measurements in cortical pyramidal neurons and with general principles of cellular Ca$^{2+}$ signaling \cite{helmchenCa2BufferingAction1996,sabatiniLifeCycleCa22002,bootmanFundamentalsCellularCalcium2020}, supporting the use of the present low-dimensional Ca$^{2+}$ model for investigating mechanosensitive channel activation and Ca$^{2+}$-dependent feedback processes.

\begin{landscape} % rotate the page
\begin{table}[ht!]
\centering
\caption{Key ion channels in cortical neurons relevant for ultrasound neuromodulation.}
\begin{tabular}{|>{\raggedright\arraybackslash}p{3cm} 
                |>{\raggedright\arraybackslash}p{3cm} 
                |>{\raggedright\arraybackslash}p{4cm} 
                |>{\raggedright\arraybackslash}p{4cm} 
                |>{\raggedright\arraybackslash}p{4cm} 
                |>{\raggedright\arraybackslash}p{4cm}|}
\hline
\textbf{Channel} & \textbf{Neuron type} & \textbf{Expression / abundance} & \textbf{Activation / Mechanosensitivity} & \textbf{Role in US modeling} & \textbf{References / Quotes} \\ \hline

TRPC1 & Pyramidal \& interneurons & Most widely distributed TRPC in the brain & Mechanosensitive, Ca$^{2+}$-permeable & Primary mechanosensitive Ca$^{2+}$ influx & ``TRPC1 is the most widely distributed member of the TRPC subfamily in the brain'' (Riccio et al., 2002; Gees et al., 2010) \\ \hline

TRPP2 (PKD2) & Pyramidal \& interneurons & Moderate & Mechanosensitive, Ca$^{2+}$-permeable, often heteromeric with TRPP1 & Primary mechanosensitive Ca$^{2+}$ influx & ``TRPP2 forms mechanosensitive Ca$^{2+}$ channels in neurons'' (González-Perrett et al., 2001; Koulen et al., 2002) \\ \hline

TRPP1 (PKD1) & Pyramidal \& interneurons & Regulatory subunit & Forms heteromers with TRPP2 & Modulates TRPP2 currents; reduces Ca$^{2+}$ conductance & ``TRPP1 modulates TRPP2 and suppresses high conductance'' (Bai et al., 2008) \\ \hline

TRPM4 & Pyramidal \& interneurons & Moderate & Ca$^{2+}$-activated nonselective cation & Amplifies depolarization after Ca$^{2+}$ influx & ``TRPM4 mediates membrane depolarization in response to intracellular Ca$^{2+}$'' (Launay et al., 2002; Guo et al., 2017) \\ \hline

Piezo1 & Pyramidal \& interneurons & Moderate & Mechanosensitive, nonselective cation & Primary mechanosensitive depolarizing current & ``Piezo channels act as primary mechanotransducers in neurons'' (Coste et al., 2010; Wu et al., 2017) \\ \hline

Piezo2 & Pyramidal \& interneurons & Lower than Piezo1 & Mechanosensitive, nonselective cation & Primary mechanosensitive depolarizing current & (Coste et al., 2010; Wu et al., 2017) \\ \hline

T-type Ca$^{2+}$ (Cav3.x) & Pyramidal \& interneurons & Moderate & Low-threshold voltage-gated & Contributes to Ca$^{2+}$ influx, amplifies subthreshold depolarizations & Huguenard \& McCormick, 1992; Cheong et al., 2001 \\ \hline

K2P (TREK1/2, TRAAK) & Pyramidal \& interneurons & High & Stretch-activated K$^+$ & Hyperpolarizing; shapes excitability & Sorum et al., 2021; Patel et al., 1998 \\ \hline

\end{tabular}
\end{table}
\end{landscape}


% ============================================================
\section{Conclusion}
% ============================================================

This model integrates nonlinear membrane mechanics with a biophysically grounded electrophysiological framework to describe ultrasound-induced mechanotransduction. Primary force-from-lipid sensors (K2P and Piezo) initiate the response, while secondary Ca$^{2+}$-dependent channels (TRP, TRPM4, T-type Ca$^{2+}$) amplify and shape membrane depolarization. Ca$^{2+}$-activated K$^+$ channels ensure recovery of the resting membrane potential.

The modeling assumptions and parameter choices are directly supported by experimental literature and provide a transparent framework for interpreting ultrasound neuromodulation experiments.

\bibliography{Ultrasound_Neuromodulation}
% \printbibliography
\end{document}
