\subsection{Intracellular Ca$^{2+}$ Dynamics and Parameter Justification}

Intracellular calcium dynamics were modeled using a reduced first-order balance
between Ca$^{2+}$ influx through Ca$^{2+}$-permeable membrane currents and Ca$^{2+}$ removal by buffering, diffusion, and active extrusion mechanisms. The temporal evolution of intracellular Ca$^{2+}$ concentration is described by \cite{helmchenSingleCompartmentModelCalcium2015}
\[
\frac{d[\mathrm{Ca}^{2+}]}{dt} = - \alpha I_{\mathrm{Ca}} - \frac{[\mathrm{Ca}^{2+}]}{\tau_{\mathrm{Ca}}} ,
\]
where $I_{\mathrm{Ca}}$ is the total Ca$^{2+}$-carrying membrane current density, $\alpha$ is a phenomenological conversion factor linking membrane current to changes in intracellular Ca$^{2+}$ concentration, and $\tau_{\mathrm{Ca}}$ is an effective Ca$^{2+}$ decay time constant representing the combined effects of buffering, diffusion, and extrusion.

This reduced formulation is obtained under the following assumptions:
(i) intracellular Ca$^{2+}$ is spatially homogeneous within the modeled
compartment, such that spatial gradients and diffusion can be lumped into a single effective decay term; (ii) Ca$^{2+}$ transients remain small compared to buffer dissociation constants, allowing buffering dynamics to be treated as fast and linear and eliminating the need to explicitly model buffer-bound Ca$^{2+}$; (iii) Ca$^{2+}$ extrusion mechanisms operate in a near-linear regime around the resting Ca$^{2+}$ concentration, so that nonlinear pump and exchanger kinetics can be approximated by a first-order relaxation with time constant $\tau_{\mathrm{Ca}}$; (iv) Ca$^{2+}$ influx is dominated by membrane Ca$^{2+}$ currents during action potentials, with other sources such as intracellular stores neglected; and (v) the resting Ca$^{2+}$ concentration is absorbed into the definition of the state variable or taken as zero without loss of generality. \cite{helmchenSingleCompartmentModelCalcium2015}

\textbf{Ca$^{2+}$ decay time constant.}
The Ca$^{2+}$ decay time constant was set to $\tau_{\mathrm{Ca}} = 50$ ms, which lies well within experimentally observed ranges for cortical pyramidal neurons. Two-photon calcium imaging studies report that somatic and hCa$^{2+}$ transients typically decay on timescales of tens of milliseconds, with reported values ranging from approximately 10 to 100 ms depending on compartment, buffer load, and pump activity.\cite{helmchenSingleCompartmentModelCalcium2015}

Helmchen et al.\ \cite{helmchenCa2BufferingAction1996} showed that Ca$^{2+}$ signals in pyramidal neuron dendrites exhibit “decay time constants of several tens of milliseconds,” while Svoboda et al.\cite{svobodaVivoDendriticCalcium1997} described Ca$^{2+}$ transients that “relax back to baseline within tens of milliseconds.” Sabatini et al. \cite{sabatiniLifeCycleCa22002} further demonstrated that endogenous buffering and extrusion mechanisms shape Ca$^{2+}$ signals with decay constants spanning this same range. More recent reviews confirm that effective Ca$^{2+}$ clearance in cortical neurons is typically “on the order of tens of milliseconds”.

Thus, a value of 50 ms represents a physiologically reasonable midpoint and is commonly employed in reduced conductance-based neuron models to capture the net effect of calcium removal mechanisms without resolving spatial diffusion explicitly \cite{destexheMechanismsUnderlyingSynchronizing1998}.

\textbf{Current-to-calcium coupling factor.}
The parameter $\alpha$ scales the contribution of Ca$^{2+}$ influx to intracellular Ca$^{2+}$ concentration changes and implicitly incorporates Faraday conversion, effective cytosolic volume, and buffering. In the present model,
$\alpha = 0.01$ was selected as a phenomenological coupling coefficient that produces Ca$^{2+}$ transients on the order of $0.1$--$1~\mu$M in response to spike-evoked and mechanically induced Ca$^{2+}$ currents.

Experimental measurements indicate that resting intracellular Ca$^{2+}$ in cortical and hippocampal pyramidal neurons is maintained in the tens of nanomolar range. For example, Helmchen et al.~\cite{helmchenCa2BufferingAction1996} reported a mean resting Ca$^{2+}$ concentration of $64 \pm 4$~nM, while other imaging studies report baseline concentrations of approximately $50$--$100$~nM depending on cellular compartment and buffering conditions \cite{svobodaVivoDendriticCalcium1997,sabatiniLifeCycleCa22002}. Single action potentials typically evoke transient Ca$^{2+}$ elevations of $\sim 0.1$--$0.5~\mu$M, whereas brief spike bursts can raise intracellular Ca$^{2+}$ into the $0.5$--$1~\mu$M range \cite{helmchenCa2BufferingAction1996,sabatiniLifeCycleCa22002}. In addition, Ca$^{2+}$ concentrations exceeding $1~\mu$M have been recorded, particularly within submembrane or nanodomain microenvironments near open Ca$^{2+}$ channels during strong synaptic activity, as well as under excessive or pathological conditions such as sustained depolarization, ischemia, or excitotoxic stress \cite{bootmanFundamentalsCellularCalcium2020}.

\textbf{Physiological consistency.}
With these parameter choices, resting intracellular Ca$^{2+}$ remains near $0.05$--$0.1~\mu$M, while electrical or mechanical stimulation induces transient Ca$^{2+}$ elevations of $0.1$--$1~\mu$M that decay on a tens-of-milliseconds timescale. Higher Ca$^{2+}$ levels ($>1~\mu$M) are interpreted as localized microdomain signals or indicators of excessive stimulation rather than steady-state cytosolic concentrations. This behavior is consistent with in vivo and in vitro calcium imaging measurements in cortical pyramidal neurons and with general principles of cellular Ca$^{2+}$ signaling \cite{helmchenCa2BufferingAction1996,sabatiniLifeCycleCa22002,bootmanFundamentalsCellularCalcium2020}, supporting the use of the present low-dimensional Ca$^{2+}$ model for investigating mechanosensitive channel activation and Ca$^{2+}$-dependent feedback processes.