\subsection{T-Type Ca$^{2+}$ Channels}

\textbf{Activation mechanism and kinetics}
T-type Ca$^{2+}$ channels are low-voltage-activated, transient calcium channels
that open in response to modest depolarizations. They are recruited following
mechanosensitive channel activation and provide additional Ca$^{2+}$ influx,
amplifying subthreshold depolarizations \cite{huguenardSimulationCurrentsInvolved1992}.

\textbf{Physiological role}
These channels act as amplifiers of subthreshold depolarizations initiated by
mechanosensitive currents. They contribute to intracellular Ca$^{2+}$ dynamics,
which can further activate Ca$^{2+}$-dependent channels such as TRPM4 and
K-Ca channels.


\textbf{Modeling formulation}  
The T-type Ca$^{2+}$ current is modeled as:
\[
I_{\mathrm{T}} = g_T \, a_T(V_m)^2 \, b_T(V_m) \, (V_m - E_{\mathrm{T}})
\]
where, $I_{\mathrm{T}}$ is the T-type Ca$^{2+}$ current (A/m$^2$), $g_T$ is the maximal channel conductance. $a_T(V_m)$ is the steady-state activation probability:
    \[
        a_T(V_m) = \frac{1}{1 + \exp(-(V_m - V_{1/2, \mathrm{act}})/k_\mathrm{act})}.
    \]
$b_T(V_m)$ is the steady-state inactivation probability:
    \[
        b_T(V_m) = \frac{1}{1 + \exp((V_m - V_{1/2, \mathrm{inact}})/k_\mathrm{inact})}.
    \]
$E_{\mathrm{T}}$ is the reversal potential for T-type channels (mV), corresponding to the Nernst potential for Ca$^{2+}$ (mV).

\textbf{Parameter justification}
\begin{itemize}
    \item The steady-state activation parameters of the T-type Ca$^{2+}$ current,
    with a half-activation voltage $V_{1/2, \mathrm{act}} = -57$~mV and slope factor of $k_\mathrm{act} = 5.6$~mV,
    were taken from Huguenard and McCormick (1992) \cite{huguenardSimulationCurrentsInvolved1992},
    consistent with experimentally measured low-threshold Ca$^{2+}$ channel
    activation in thalamic and other central neurons.

    \item The steady-state inactivation parameters, with a half-inactivation
    voltage $V_{1/2, \mathrm{inact}} = -80$~mV and slope factor of $k_\mathrm{inact} = 4.0$~mV, reproduce the rapid
    voltage-dependent inactivation characteristic of T-type Ca$^{2+}$ channels,
    as reported experimentally and implemented in \cite{huguenardSimulationCurrentsInvolved1992}.

    \item Single-channel recordings under physiological ionic conditions report
    a unitary conductance in the range of approximately 5–9~pS for T-type
    Ca$^{2+}$ channels, which was used to guide the choice of maximal conductance
    in the model \cite{huguenardSimulationCurrentsInvolved1992}.

    \item All T-type Ca$^{2+}$ current was included in the intracellular Ca$^{2+}$
    dynamics, as these channels are highly Ca$^{2+}$ selective and represent a
    major source of activity-dependent Ca$^{2+}$ influx. This modeling assumption
    follows the approach of Huguenard and McCormick (1992), where Ca$^{2+}$ entry
    through T-type channels directly contributes to intracellular Ca$^{2+}$
    accumulation and downstream Ca$^{2+}$-dependent processes. \cite{huguenardSimulationCurrentsInvolved1992}
\end{itemize}


\textbf{Activation and inactivation kinetics}
T-type channels activate rapidly upon depolarization and inactivate over tens of milliseconds,
providing transient Ca$^{2+}$ influx. Due to their fast kinetics relative to slower
mechanosensitive currents, explicit gating variables can be omitted in some
simplified models, but in this implementation, steady-state activation and
inactivation probabilities ($a_T$, $b_T$) are used to capture the transient response
accurately.