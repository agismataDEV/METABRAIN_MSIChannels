\subsection{K2P Potassium Channels (TRAAK, TREK-1, TREK-2)}

Two-pore-domain potassium (K2P) channels are among the best-characterized
force-from-lipid mechanosensitive channels that contribute to action potential propagation, sensory transduction, and muscle contraction. Structural and functional studies demonstrate that TRAAK, TREK-1, and TREK-2 open in response to increased in-plane membrane tension without requiring auxiliary proteins or cytoskeletal coupling \cite{brohawnMechanosensitivityMediatedDirectly2014,sorumUltrasoundActivatesMechanosensitive2021, sorumTensionActivationMechanosensitive2024}.

\textbf{Activation mechanism and kinetics}
K2P channels such as TRAAK, TREK-1, and TREK-2 are activated directly by membrane
tension via a force-from-lipid mechanism. Channel opening occurs when increased
in-plane membrane tension stabilizes the open conformation by expanding the
effective gate area embedded in the lipid bilayer. Their gating is regulated by mechanical perturbation of the cell membrane as well as polyunsaturated fatty acids, other lipids, and temperature. \cite{brohawnMechanosensitivityMediatedDirectly2014}

Activation is rapid, occurring on sub-millisecond to millisecond timescales
($\sim 0.1$--$1$ ms) following the onset of membrane stretch. Importantly, K2P
channels do not exhibit intrinsic inactivation on physiological timescales.
Instead, deactivation occurs promptly when membrane tension falls below the
activation threshold, with closure times comparable to activation times
\cite{sorumUltrasoundActivatesMechanosensitive2021}. A consequence of the rapid ultrasonic activation of TRAAK is that even brief stimulation can activate large currents: 0.15~ms and 0.8~ms stimulation result in approximately 50\% and 95\% of the maximal TRAAK current, respectively~\cite{sorumUltrasoundActivatesMechanosensitive2021}.

\textbf{Physiological role}
The K2P current is a fast, non-inactivating, mechanosensitive outward K$^+$ current directly controlled by membrane tension. Upon activation, K2P channels drive the membrane potential toward $E_K$, thereby stabilizing and repolarizing the membrane and providing rapid negative feedback during ultrasound-induced depolarization.

\textbf{Ions conducted}
Primarily K$^+$.

\textbf{Modeling formulation}  
\[
I_{\mathrm{K2P}} = g_{\mathrm{K2P}} \, P_0(t) \, (V_m - E_K2P)
\]
where, $I_{\mathrm{K2P}}$ is the current carried by mechanosensitive two-pore-domain potassium (K2P) channels (e.g., TRAAK, TREK-1, TREK-2). $g_{\mathrm{K2P}}$ is the maximal K2P conductance per unit membrane area (S\,m$^{-2}$). $P_0(t)$ is the time-dependent open probability of the channel, determined by membrane tension induced by ultrasound-driven deformation. $V_m$ is the membrane potential (mV) and $E_K2P$ is the reversal potential. 

\textbf{Parameter justification}
Values for these parameters were chosen based on previous research studies.
$E_K2P$ was given a value of equal to the reversal potential of the Potassium reversal (Nernst) potential (mV). Sorum et al. \cite{sorumUltrasoundActivatesMechanosensitive2021} mention a reverse potential close to $E_{K^+}$, with a value ~ -75 mV for TRAAK. Patel et al. \cite{patelMammalianTwoPore1998} mention a reversal potential at the predicted value of K+ equillibrium potential at -83 mV. It is assumed in our study that it is equal to the Potassium Nerst potential.

The open probability is computed from Eq.~\ref{eq:MSC_Prob} and Sorum et al. \cite{sorumTensionActivationMechanosensitive2024} found that $T_{1/2}$ is 4.4 mN/m , 6.4 mN/m and 5.8 mN/m and the slope is 1.7, 2.3 and 1.4 while the change in area is 2.4 nm$^2$, 1.8 nm$^2$, 2.9 nm$^2$ for TRAAK, TREK-1 and TREK-2, respectively.

Unitary conductances were found to be 73~pS~\cite{sorumTensionActivationMechanosensitive2024} and 65~pS at 100~mV~\cite{blinMixingMatchingTREK2016} for TRAAK; 48~pS at 50~mV~\cite{patelMammalianTwoPore1998} and 88~pS at 100~mV~\cite{blinMixingMatchingTREK2016} for TREK-1; and 40~pS at 100~mV~\cite{blinMixingMatchingTREK2016} for TREK-2.

\cite{Expression}
The K2P family channels are highly expressed in the brain, particularly in the hippocampus and cortex, and are found in both pyramidal cortical neurons and GABAergic interneurons \cite{djillaniRoleTREK1Health2019}. They are also strongly enriched at nodes of Ranvier, with densities approximately 3,000 times higher than in somatic regions \cite{kandaTREK1TRAAKAre2019}.